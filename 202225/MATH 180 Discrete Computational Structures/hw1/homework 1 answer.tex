\documentclass[11pt, oneside]{article}   	% use "amsart" instead of "article" for AMSLaTeX format
\usepackage{geometry}                		% See geometry.pdf to learn the layout options. There are lots.
\geometry{letterpaper}                   		% ... or a4paper or a5paper or ... 
%\geometry{landscape}                		% Activate for rotated page geometry
\usepackage[parfill]{parskip}    			% Activate to begin paragraphs with an empty line rather than an indent
\usepackage{graphicx}				% Use pdf, png, jpg, or eps§ with pdflatex; use eps in DVI mode
								% TeX will automatically convert eps --> pdf in pdflatex		
\usepackage{amssymb}
\usepackage{mathtools}
\usepackage{enumerate}
\usepackage{tikz}

\usetikzlibrary{arrows}

\def\firstcircle{(90:1.75cm) circle (2.5cm)}
\def\secondcircle{(210:1.75cm) circle (2.5cm)}
\def\thirdcircle{(330:1.75cm) circle (2.5cm)}


%SetFonts

%SetFonts


\title{\bf MATH 180 - Homework 1}
\author{Lixiao Yang - \texttt{ly364@drexel.edu}}
\date{January 15, 2022}							% Activate to display a given date or no date

\begin{document}

\maketitle

\section*{Question 1}

	\begin{enumerate}
		\item \{5, 7\}
		\item \{1, 3, 4, 5, 6, 7, 8, 11\}
		\item \{1, 3, 11\} 
		\item \{4, 6, 8\} 
	\end{enumerate}

\section*{Question 2}

	\begin{enumerate}
		\item 27
		\item 2
		\item 13
	\end{enumerate}

\section*{Question 3}

	\{2, 4, 6\} 

\section*{Question 4}

	\{1, 2, 3, 4, 5, 6, 7, 8, 10, 12\} 

\section*{Question 5}

	A = \{1, 2, 3\} , B = \{2, 3, 4, 5\} 

\section*{Question 6}

	Since $\left\lvert A \cup B \right\rvert = \left\lvert A \right\rvert + \left\lvert B\right\rvert  - \left\lvert A\cap B\right\rvert$, 
	with the conditions listed above, we can get that $\left\lvert A\right\rvert  + \left\lvert B\right\rvert  = 15$.\\
	With $\left\lvert A\right\rvert  = \left\lvert B\right\rvert $ we can conclude that $\left\lvert A\right\rvert  = \left\lvert B\right\rvert = 7.5$
	which is impossible for a set.\\ So there are no conforming sets A and B.

\section*{Question 7}

	\begin{enumerate}
		\item The smallest possible value is 4, the largest possible value is 9.
		\item The smallest possible value is 0, the largest possible value is 4.
		\item The smallest possible value is 4, the largest possible value is 20.
	\end{enumerate}

\section*{Question 8}

	\begin{enumerate}
		\item $A = \{11, 13, 15, 16, 17, 18, 19, 20, 21, 22\}$\\This set has a size 10 and it fits the result that $X \setminus A = \{10, 12, 14\}$.
		\item $B = \{n\in \mathbb{N} : 10 \leq n \leq 14\}$\\This set has a size 5 and satisfies $B \subseteq X$.
		\item $E = \{10, 10, 10, 10, 10, 10, 10, 10, 10, 10\}$\\This set is a subset of $X$ and $\left\lvert E\right\rvert = E = 10$.
	\end{enumerate}


\end{document}  