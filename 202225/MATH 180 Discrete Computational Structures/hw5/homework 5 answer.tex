\documentclass[11pt, oneside]{article}   	% use "amsart" instead of "article" for AMSLaTeX format
\usepackage{geometry}                		% See geometry.pdf to learn the layout options. There are lots.
\geometry{letterpaper}                   		% ... or a4paper or a5paper or ... 
%\geometry{landscape}                		% Activate for rotated page geometry
\usepackage[parfill]{parskip}    			% Activate to begin paragraphs with an empty line rather than an indent
\usepackage{graphicx}				% Use pdf, png, jpg, or eps§ with pdflatex; use eps in DVI mode
								% TeX will automatically convert eps --> pdf in pdflatex		
\usepackage{amssymb}
\usepackage{amsmath, amsfonts}
\usepackage{mathtools}
\usepackage{enumerate}
\usepackage{tikz}

\usetikzlibrary{arrows}

\def\firstcircle{(90:1.75cm) circle (2.5cm)}
\def\secondcircle{(210:1.75cm) circle (2.5cm)}
\def\thirdcircle{(330:1.75cm) circle (2.5cm)}


%SetFonts

%SetFonts


\title{\bf MATH 180 - Homework 5}
\author{Lixiao Yang - \texttt{ly364@drexel.edu}}
\date{February 14, 2023}							% Activate to display a given date or no date

\begin{document}

\maketitle

\section*{Question 1}

	First differences: ${1,3,5,7,9,11}$\\
	Second differences: ${2,2,2,2,2}$\\
	$\therefore$ The sequence is $\Delta^{2}$-constant.\\
	Let $a_{n}=an^{2}+bn+c$, since $a_{0}=4, a_{1}=5, a_{2}=8$, we have \\
	\begin{equation}
		\left \{
			\begin{aligned}
			c=4\\
			a+b+c=5\\
			4a+2b+c=8
			\end{aligned}
		\right .
	\end{equation}
	$\therefore a=1,b=0,c=4$ and the formula is $a_{n}=n^{2}+4$.

\section*{Question 2}

	First differences: ${3,4,5,6,7,8}$\\
	Second differences: ${1,1,1,1,1}$\\
	$\therefore$ The sequence is $\Delta^{2}$-constant.\\
	Let $a_{n}=an^{2}+bn+c$, since $a_{0}=2, a_{1}=5, a_{2}=9$, we have \\
	\begin{equation}
		\left \{
			\begin{aligned}
			c=2\\
			a+b+c=5\\
			4a+2b+c=9
			\end{aligned}
		\right .
	\end{equation}
	$\therefore a=\frac{1}{2},b=\frac{5}{2},c=2$ and the formula is $a_{n}=\frac{1}{2}n^{2}+\frac{5}{2}n+2$.

\section*{Question 3}

	First differences: ${2,3,7,14,24,37}$\\
	Second differences: ${1,4,7,10,13}$\\
	Third differences: ${3,3,3,3}$\\
	$\therefore$ The sequence is $\Delta^{3}$-constant.\\
	Let $a_{n}=an^{3}+bn^{2}+cn+d$, since $a_{0}=0, a_{1}=2, a_{2}=5, a_{3}=12$, we have \\
	\begin{equation}
		\left \{
			\begin{aligned}
			d=0\\
			a+b+c+d=2\\
			8a+4b+2c+d=5\\
			27a+9b+3c+d=12
			\end{aligned}
		\right .
	\end{equation}
	$\therefore a=\frac{1}{2},b=-1,c=\frac{5}{2},d=0$ and the formula is $a_{n}=\frac{1}{2}n^{3}-n^{2}+\frac{5}{2}n$.

\section*{Question 4}

	\subsection*{Part a}
	$P(3)=10, P(4)=20,P(5)=35$
	\subsection*{Part b}
	Sequence: 1,4,10,20,35,56,\ldots \\
	First differences: ${3,6,10,15,21}$\\
	Second differences: ${3,4,5,6}$\\
	Third differences: ${1,1,1}$\\
	$\therefore$ The sequence is $\Delta^{3}$-constant.\\
	Let $P(n)=an^{3}+bn^{2}+cn+d$, since $P(0)=0, P(1)=1, P(2)=4, P(3)=10$, we have \\
	\begin{equation}
		\left \{
			\begin{aligned}
			d=0\\
			a+b+c+d=1\\
			8a+4b+2c+d=4\\
			27a+9b+3c+d=10
			\end{aligned}
		\right .
	\end{equation}
	$\therefore a=\frac{1}{6},b=\frac{1}{2},c=\frac{1}{3},d=0$ and the formula is $P(n)=\frac{1}{6}n^{3}+\frac{1}{2}n^{2}+\frac{1}{3}n$.
	\subsection*{Part c}
	$P(15)=680$
	\subsection*{Part d}
	For every layer added, the number of cannonballs are the former layer number adds the current layer's number.

\section*{Question 5}

	$\because a_{n}-a_{n-1}=2^{n}$ with $a_{0}=3$.\\
	By iteration, we can get $a_{n}-a_{0}=2+\cdots 2^{n}$.\\
	$\therefore a_{n}=3+2+2^{2}+\cdots 2^{n}$ so that $-2a_{n}=-6-2^{2}-2^{3}-\cdots -2^{n+1}$\\
	$\therefore a_{n}=2^{n+1}+1$

\section*{Question 6}

	Let $a_{n}=r^{n}$, we have $r^{n}=5r^{n-1}+6r^{n-2}$\\
	$\therefore r^{2}-5r-6=0$\\
	By solving the equation we have $r=6, r=-1$.\\
	$\therefore a_{n}$ can be written as $a_{n}=A6^{n}+B(-1)^{n}$\\
	$\because a_{0}=1,a_{1}=13$\\
	$\therefore$ We can get $A=2,B=-1$\\
	$\therefore a_{n}=2\cdot 6^{n}+(-1)^{n+1}$

\section*{Question 7}

	Let $a_{n}=r^{n}$, we have $r^{n}=5r^{n-1}-6r^{n-2}$\\
	$\therefore r^{2}-5r+6=0$\\
	By solving the equation we have $r=2, r=3$.\\
	$\therefore a_{n}$ can be written as $a_{n}=A\cdot 2^{n}+B\cdot3^{n}$\\
	$\because a_{0}=4,a_{1}=11$\\
	$\therefore$ We can get $A=1,B=3$\\
	$\therefore a_{n}=2^{n}+3^{n+1}$

\section*{Question 8}

	Let $a_{n}=r^{n}$, we have $r^{n}=8r^{n-1}-16r^{n-2}$\\
	$\therefore r^{2}-8r+16=0$\\
	By solving the equation we have $r=4$.\\
	$\therefore a_{n}$ can be written as $a_{n}=A\cdot4^{n}+Bn\cdot4^{n}$\\
	$\because a_{0}=3,a_{1}=10$\\
	$\therefore$ We can get $A=3,B=-\frac{1}{2}$\\
	$\therefore a_{n}=3\cdot4^{n}-\frac{n}{2}\cdot4^{n}$

\section*{Question 9}

	\subsection*{Part a}
	Recursive formula: $a_{n}=4a_{n-1}$ with $a_{0}=1$\\
	Closed formula: $a_{n}=4^{n}$\\
	\subsection*{Part b}
	Let $a_{n}=r^{n}$, we have $r^{n}=4r^{n-1}$\\
	$\therefore r-4=0$\\
	By solving the equation we have $r=4$.\\
	$\therefore a_{n}$ can be written as $a_{n}=4^{n}$


\end{document}  