\documentclass[11pt, oneside]{article}   	% use "amsart" instead of "article" for AMSLaTeX format
\usepackage{geometry}                		% See geometry.pdf to learn the layout options. There are lots.
\geometry{letterpaper}                   		% ... or a4paper or a5paper or ... 
%\geometry{landscape}                		% Activate for rotated page geometry
\usepackage[parfill]{parskip}    			% Activate to begin paragraphs with an empty line rather than an indent
\usepackage{graphicx}				% Use pdf, png, jpg, or eps§ with pdflatex; use eps in DVI mode
								% TeX will automatically convert eps --> pdf in pdflatex		
\usepackage{amssymb}
\usepackage{amsmath, amsfonts}
\usepackage{mathtools}
\usepackage{enumerate}
\usepackage{tikz}

\usetikzlibrary{arrows}

\def\firstcircle{(90:1.75cm) circle (2.5cm)}
\def\secondcircle{(210:1.75cm) circle (2.5cm)}
\def\thirdcircle{(330:1.75cm) circle (2.5cm)}


%SetFonts

%SetFonts


\title{\bf MATH 180 - Homework 6}
\author{Lixiao Yang - \texttt{ly364@drexel.edu}}
\date{February 28, 2023}							% Activate to display a given date or no date

\begin{document}

\maketitle

\section*{Question 1}

	For P(0): $0-0$ is obviously divisible by 3.\\
	For P(k): Let $k^{3}-k=3m$\\
	$\therefore (k+1)^3-(k+1)=k^{3}+3k^{2}+3k+1-k-1=3m+3(k^{2}+k)=3(k^{2}+k+m)$\\
	$\therefore P(k)\rightarrow P(k+1)$\\
	$\because P(0)$ is true\\
	$\therefore$ P(n) is true for $\forall n\in \mathbb{N}$. 

\section*{Question 2}

	For P(4): $4!=24>16=2^{4}$.\\
	For P(k): Let $k!>2^{n}$\\
	$\therefore (k+1)!=(k+1)k!>(k+1)2^{k}$\\
	$\because k+1\geq 5$\\
	$\therefore (k+1)2^{k}>4\cdot2^{k}=2^{k+2}>2^{k+1}$\\
	$\therefore P(k)\rightarrow P(k+1)$\\
	$\because P(0)$ is true\\
	$\therefore$ P(n) is true for $\forall n\in \mathbb{N}$ with $n\geq 4$.

\section*{Question 3}

	For P(2): $2!=2<2^{2}=4$.\\
	For P(k): Let $k!<k^{k}$\\
	$\therefore (k+1)!=(k+1)k!<(k+1)k^{k}$
	$\because k\geq 2$\\
	$\therefore (k+1)k^{k}<(k+1)^{k+1}$\\
	$\therefore (k+1)!<(k+1)^{k+1}$\\
	$\because P(0)$ is true\\
	$\therefore$ P(n) is true for $\forall n\in \mathbb{N}$ with $n>4$. 

\section*{Question 4}

	P(1) is not true.\\
	Let the two sets be $S_{1},S_{2}$, the horse color in each set is same. However, there are no overlap
	in the two sets when $n=1$. Thus we can not conclude that $S_{1}$ has the same color as $S_{2}$.


\end{document}  