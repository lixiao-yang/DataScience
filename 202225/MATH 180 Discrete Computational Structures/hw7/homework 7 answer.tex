\documentclass[11pt, oneside]{article}   	% use "amsart" instead of "article" for AMSLaTeX format
\usepackage{geometry}                		% See geometry.pdf to learn the layout options. There are lots.
\geometry{letterpaper}                   		% ... or a4paper or a5paper or ... 
%\geometry{landscape}                		% Activate for rotated page geometry
\usepackage[parfill]{parskip}    			% Activate to begin paragraphs with an empty line rather than an indent
\usepackage{graphicx}				% Use pdf, png, jpg, or eps§ with pdflatex; use eps in DVI mode
								% TeX will automatically convert eps --> pdf in pdflatex		
\usepackage{amssymb}
\usepackage{amsmath, amsfonts}
\usepackage{mathtools}
\usepackage{enumerate}
\usepackage{tikz}

\usetikzlibrary{arrows}

\def\firstcircle{(90:1.75cm) circle (2.5cm)}
\def\secondcircle{(210:1.75cm) circle (2.5cm)}
\def\thirdcircle{(330:1.75cm) circle (2.5cm)}


%SetFonts

%SetFonts


\title{\bf MATH 180 - Homework 7}
\author{Lixiao Yang - \texttt{ly364@drexel.edu}}
\date{March 10, 2023}							% Activate to display a given date or no date

\begin{document}

\maketitle

\section*{Question 1}

	\begin{tabular}{cccccc}
	 \hline
	 $x$ & $y$ & $x\Leftrightarrow y$ & $x\Rightarrow y$ & $y\Rightarrow x$ & $(x\Rightarrow y)\wedge (y\Rightarrow x)$  \\
	 \hline
	 T & T & T & T & T & T \\
	 T & F & F & F & T & F \\
	 F & T & F & T & F & F \\
	 F & F & T & T & T & T \\
	 \hline
	\end{tabular}

\section*{Question 2}

	\begin{tabular}{ccccccc}
	 \hline
	 $x$ & $y$ & $z$ & $(x\vee y)\Leftrightarrow z$ & $x\Rightarrow z$ & $y\Rightarrow z$ & $(x\Rightarrow z)\wedge (y\Rightarrow z)$  \\
	 \hline
	 T & T & T & T & T & T & T \\
	 T & T & F & F & F & F & F \\
	 T & F & T & T & T & T & T \\
	 T & F & F & F & F & T & F \\
	 F & T & T & T & T & T & T \\
	 F & T & F & F & T & F & F \\
	 F & F & T & T & T & T & T \\
	 F & F & F & T & T & T & T \\
	 \hline
	\end{tabular}

\section*{Question 3}

\begin{tabular}{ccccc}
	\hline
	$x$ & $y$ & $x\Rightarrow y$ & $x\wedge (x\Rightarrow y)$ & $(x\wedge (x\Rightarrow y))\Rightarrow y$  \\
	\hline
	T & T & T & T & T \\
	T & F & F & F & T \\
	F & T & T & F & T \\
	F & F & T & F & T \\
	\hline
   \end{tabular} 

\section*{Question 4}

   Let $m,n$ be an interger\\
   $\therefore$ $x, y$ can be written as $x=2m+1, y=2n$.\\
   $\therefore x+y=2m+2n+1=2(m+n)+1$\\
   $\therefore$ For $x,y\in \mathbb{Z}, x+y$ is an odd integer.

\section*{Question 5}

   Let $m,n$ be an interger\\
   $\therefore$ $x, y$ can be written as $x=2m+1, y=2n+1$.\\
   $\therefore x\times y=(2m+1)(2n+1)=4mn+2m+2n+1=2(2mn+m+n)+1$\\
   $\therefore$ The product of two odd integers is odd.

\section*{Question 6}

   Let $k\in \mathbb{Z} $\\
   $\therefore$ $a|b$ can be written as $b=ka$\\
   $\therefore bc=(ka)c=a(kc)$\\
   $\therefore$ If $a|b$ then $a|bc$.\\

\section*{Question 7}

   Let $j,k\in \mathbb{Z} $\\
   $\therefore b=ja, d=kc$\\
   $\therefore bd=(ja)(kc)=jk(ac)$\\
   $\therefore$ If $a|b$ and $c|d$, then $ac|bd$.

\section*{Question 8}

   Let $n=m+k$ for $m^{2}, n^{2}$.\\
   $\therefore n^{2}-m^{2}=2mk+k^{2}$ for some $k\geqslant 2$.\\
   $\because 2mk+k^{2}=k(2m+k)$\\
   $\therefore$ The difference between distinct, nonconsecutive perfect squares is composite.

\section*{Question 9}

   The statement is equal to: $(a, b\in \mathbb{R})\wedge (ab=0)\Rightarrow (a=0)\vee (b=0)$\\
   Its contradiction: $\lnot (a, b\in \mathbb{R})\vee \lnot(ab=0)\Rightarrow (a=0)\vee (b=0)$.\\
   If $\lnot (a, b\in \mathbb{R})$, then $a, b\neq 0$.\\
   If $ab\neq 0$, then $a,b\neq 0$.\\
   So the contradiction is false, the statement is true.

\section*{Question 10}

   \subsection*{Part a}
   Suppose not. Then $\sqrt{3}$ is equal to a fraction $\frac{a}{b}$.\\
   Without loss of generality, assume $\frac{a}{b}$ is in lowest terms. So,\\
   $$3=\frac{a^{2}}{b^{2}}$$\\
   $$3b^{2}=a^{2}$$\\
   Thus $a^{2}$ is divisible by 3. So $a=3k$ for some integer k, and $a^{2}=9k^{2}$. We then have,\\
   $3b^{2}=9k^{2}, b^{2}=3k^{2}$ which is a contradiction.\\
   $\therefore \sqrt{3}$ is irrational.

   \subsection*{Part b}
   Suppose not. Then $\sqrt[3]{2}$ is equal to a fraction $\frac{a}{b}$.\\
   Without loss of generality, assume $\frac{a}{b}$ is in lowest terms. So,\\
   $$2=\frac{a^{3}}{b^{3}}$$\\
   $$2b^{3}=a^{3}$$\\
   Thus $a^{3}$ is even. So $a=2k$ for some integer $k$, and $a^{3}=8k^{3}$. We then have,\\
   $2b^{3}=8k^{3}, b^{3}=2k^{3}$ which is a contradiction.\\
   $\therefore \sqrt[3]{2}$ is irrational.

\end{document}  